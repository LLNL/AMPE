\documentclass[12pt]{letter}
\setlength{\textheight}{9.0in}
\setlength{\textwidth}{6.0in}
\setlength{\topmargin}{-0.4in}

\begin{document}

\pagestyle{empty}

\begin{letter}

\vspace{0.2in}
\centerline{\bf Code Abstract for KINSOL v2.2.0}

\begin{enumerate}

\item {\bf Identification}

Software acronym: KINSOL v2.2.0

Short title: Nonlinear algebraic system solver.

\item {\bf Developer name(s) and affiliations}

Alan C. Hindmarsh, Radu Serban, and Aaron M. Collier (CASC, LLNL)

\item {\bf Software Completion Date}

November 19, 2004

\item {\bf Brief description}

KINSOL is a general purpose (serial and parallel) solver for nonlinear algebraic 
systems which can be described as F(u) = 0. It is based on  Newton-Krylov
solver technology.

\item {\bf Method of solution}

Solution is by an inexact Newton method. For the solution of linear systems within 
Newton iteration, only a preconditioned GMRES solver is provided at this time.
This provides a linear solver is applied in a matrix-free manner, with matrix-vector 
products obtained by either finite difference quotients or a user-supplied routine.

\item {\bf Computer(s) for which software is written}

KINSOL should run on any computer with an ANSI C compiler. The appropriate 
precision (single, double, or extended) is selected at the configuration phase.

For use with Fortran applications, a set of Fortran/C interface routines,
called FKINSOL, is also supplied.  These are written in C, but assume that
the user calling program and all user-supplied routines are in Fortran.

\item {\bf Operating system}

No system-dependency in the software itself. But installation is system-dependent. 
The package supplied consists of a single archived file. Installation from this file 
assumes a UNIX system with the tar utility. Configuration is done through a 
configure script. Compilation of libraries is done by way of UNIX makefiles.

\item {\bf Programming language(s) used}

ANSI C (100\%)

\item {\bf Software limitations}

none

\item {\bf Unique features of the software}

KINSOL is organized in a highly modular manner. The basic nonlinear solver
is separate from, and independent of, the linear system 
solver, as well as the vector operation modules. Thus the set of linear solvers can be 
expanded and the internal vector representation can be replaced with no impact on 
the main solver.


\item {\bf Related and auxiliary software}

KINSOL is part of SUNDIALS (Suite of Nonlinear and Differential/Algebraic equation 
Solvers). 

\item {\bf Other Programming or Operating Information or Restrictions}

none


\item {\bf Hardware Requirements}

none


\item {\bf Time Requirements}

Timing is highly dependent on machine and problem.


\item {\bf References}

Document provided with the distribution
\begin{itemize}
\item A. M. Collier, A. C. Hindmarsh, R. Serban, and C. S. Woodward,
    "User Documentation for KINSOL v2.2.0," LLNL technical report
    UCRL-SM-208116, November 2004. 
\item A. M. Collier and R. Serban, "Example Programs for KINSOL v2.2.0,"
    LLNL technical report UCRL-SM-208114, November 2004.
\end{itemize}
Additional background references
\begin{itemize}
\item A. C. Hindmarsh, P. N. Brown, K. E. Grant, S. L. Lee, R. Serban, 
    D. E. Shumaker, and C. S. Woodward, "SUNDIALS, Suite of Nonlinear and 
    Differential/Algebraic Equation Solvers," ACM Trans. Math. Softw., 
    accepted, 2004.
\item Peter N. Brown and Youcef Saad, "Hybrid Krylov Methods for
    Nonlinear Systems of Equations," SIAM J. Sci. Stat. Comput., 
    Vol 11, no 3, pp. 450-481, May 1990.  
\end{itemize}
\end{enumerate}

\end{letter}
\end{document}