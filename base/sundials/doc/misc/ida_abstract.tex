\documentclass[12pt]{letter}
\setlength{\textheight}{9.0in}
\setlength{\textwidth}{6.0in}
\setlength{\topmargin}{-0.4in}

\begin{document}

\pagestyle{empty}

\begin{letter}

\vspace{0.2in}
\centerline{\bf Code Abstract for IDA v2.2.0}

\begin{enumerate}

\item {\bf Identification}

Software acronym: IDA v2.2.0

Short title: Stiff DAE integrator.

\item {\bf Developer name(s) and affiliations}

Alan C. Hindmarsh and Radu Serban (CASC, LLNL)

\item {\bf Software Completion Date}

November 19, 2004

\item {\bf Brief description}

IDA is a general purpose (serial and parallel) solver for differential algebraic
equation (DAE) systems or implicit ordinary differential equation (ODE) systems.

\item {\bf Method of solution}

Integration is by the BDF method. Corrector iteration is by Newton iteration. 
For the solution of linear systems within Newton iteration, users can select a 
dense solver, a band solver, or a preconditioned GMRES solver.

\item {\bf Computer(s) for which software is written}

IDA should run on any computer with an ANSI C compiler. The appropriate 
precision (single, double, or extended) is selected at the configuration phase.

\item {\bf Operating system}

No system-dependency in the software itself. But installation is system-dependent. 
The package supplied consists of a single archived file. Installation from this file 
assumes a UNIX system with the tar utility. Configuration is done through a 
configure script. Compilation of libraries is done by way of UNIX makefiles.

\item {\bf Programming language(s) used}

ANSI C (100\%)

\item {\bf Software limitations}

none

\item {\bf Unique features of the software}

IDA is organized in a highly modular manner. The basic integrator
is separate from, and independent of, the linear system 
solvers, as well as the vector operation modules. Thus the set of linear solvers can be 
expanded and the internal vector representation can be replaced with no impact on 
the main solver.

\item {\bf Related and auxiliary software}

IDA is part of SUNDIALS (Suite of Nonlinear and Differential/Algebraic equation 
Solvers). 

\item {\bf Other Programming or Operating Information or Restrictions}

none


\item {\bf Hardware Requirements}

none


\item {\bf Time Requirements}

Timing is highly dependent on machine and problem.


\item {\bf References}

Document provided with the distribution
\begin{itemize}
\item A. C. Hindmarsh and R. Serban, "User Documentation for IDA v2.2.0," 
    LLNL technical report UCRL-SM-208112, November 2004.
\item A. C. Hindmarsh and R. Serban, "Example Programs for IDA v2.2.0," 
    LLNL technical report UCRL-SM-208113, November 2004.
\end{itemize}
Additional background references
\begin{itemize}
\item A. C. Hindmarsh, P. N. Brown, K. E. Grant, S. L. Lee, R. Serban, 
    D. E. Shumaker, and C. S. Woodward, "SUNDIALS, Suite of Nonlinear and 
    Differential/Algebraic Equation Solvers," ACM Trans. Math. Softw., 
    accepted, 2004.
\item P. N. Brown, A. C. Hindmarsh, and L. R. Petzold, Using Krylov Methods 
    in the Solution of Large-Scale Differential-Algebraic Systems, 
    SIAM J. Sci. Comp., 15 (1994), pp. 1467-1488.
\item P. N. Brown, A. C. Hindmarsh, and L. R. Petzold, Consistent Initial 
    Condition Calculation for Differential-Algebraic Systems, 
    SIAM J. Sci. Comp., 19 (1998), pp. 1495-1512.
\end{itemize}
\end{enumerate}

\end{letter}
\end{document}