%% foo.tex
\begin{samepage}
\hrule
\begin{center}
\phantomsection
{\large \verb!mpistart!}
\label{p:mpistart}
\index{mpistart}
\end{center}
\hrule\vspace{0.1in}

%% one line -------------------

\noindent{\bf \sc Purpose}

\begin{alltt}
MPISTART invokes lamboot (if required) and MPI_Init (if required).
\end{alltt}

\end{samepage}


%% definition  -------------------

\begin{samepage}

\noindent{\bf \sc Synopsis}

\begin{alltt}
function mpistart(nslaves, rpi, hosts) 
\end{alltt}

\end{samepage}

%% description -------------------

\noindent{\bf \sc Description}

\begin{alltt}
MPISTART invokes lamboot (if required) and MPI_Init (if required).

   Usage: MPISTART [ ( NSLAVES [, RPI [, HOSTS] ] ) ]

   MPISTART boots LAM and initializes MPI to match a given number of slave 
   hosts (and rpi) from a given list of hosts. All three args optional.

   If they are not defined, HOSTS are taken from a builtin HOSTS list
   (edit HOSTS at the beginning of this file to match your cluster)
   or from the bhost file if defined through LAMBHOST (in this order).

   If not defined, RPI is taken from the builtin variable RPI (edit it
   to suit your needs) or from the LAM_MPI_SSI_rpi environment variable
   (in this order).
\end{alltt}






\vspace{0.1in}