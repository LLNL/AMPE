%% foo.tex
\begin{samepage}
\hrule
\begin{center}
\phantomsection
{\large \verb!N_VMaxNorm!}
\label{p:N_VMaxNorm}
\index{N_VMaxNorm}
\end{center}
\hrule\vspace{0.1in}

%% one line -------------------

\noindent{\bf \sc Purpose}

\begin{alltt}
N_VMaxNorm returns the L-infinity norm of x
\end{alltt}

\end{samepage}


%% definition  -------------------

\begin{samepage}

\noindent{\bf \sc Synopsis}

\begin{alltt}
function ret = N_VMaxNorm(x, comm) 
\end{alltt}

\end{samepage}

%% description -------------------

\noindent{\bf \sc Description}

\begin{alltt}
N_VMaxNorm returns the L-infinity norm of x

   Usage:  RET = N_VMaxNorm ( X [, COMM] )

If COMM is not present, N_VMaxNorm returns the L-infinity norm 
of the local portion of X. Otherwise, it returns the global
L-infinity norm..
\end{alltt}





 
%% source -------------------

\noindent{\bf \sc Source Code}

\input{nvector/N_VMaxNormsrc}
\vspace{0.1in}