%% foo.tex
\begin{samepage}
\hrule
\begin{center}
\phantomsection
{\large \verb!IDAReInit!}
\label{p:IDAReInit}
\index{IDAReInit}
\end{center}
\hrule\vspace{0.1in}

%% one line -------------------

\noindent{\bf \sc Purpose}

\begin{alltt}
IDAReInit reinitializes memory for IDAS.
\end{alltt}

\end{samepage}


%% definition  -------------------

\begin{samepage}

\noindent{\bf \sc Synopsis}

\begin{alltt}
function [] = IDAReInit(t0,yy0,yp0,options) 
\end{alltt}

\end{samepage}

%% description -------------------

\noindent{\bf \sc Description}

\begin{alltt}
IDAReInit reinitializes memory for IDAS.
   where a prior call to IDAInit has been made with the same
   problem size N. IDAReInit performs the same input checking
   and initializations that IDAInit does, but it does no 
   memory allocation, assuming that the existing internal memory 
   is sufficient for the new problem.

   Usage: IDAReInit ( T0, YY0, YP0 [, OPTIONS ] ) 

   T0       is the initial value of t.
   YY0      is the initial condition vector y(t0).  
   YP0      is the initial condition vector y'(t0).  
   OPTIONS  is an (optional) set of integration options, created with
            the IDASetOptions function. 

  See also: IDASetOptions, IDAInit
\end{alltt}






\vspace{0.1in}