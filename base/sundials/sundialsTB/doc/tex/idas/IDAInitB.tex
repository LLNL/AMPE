%% foo.tex
\begin{samepage}
\hrule
\begin{center}
\phantomsection
{\large \verb!IDAInitB!}
\label{p:IDAInitB}
\index{IDAInitB}
\end{center}
\hrule\vspace{0.1in}

%% one line -------------------

\noindent{\bf \sc Purpose}

\begin{alltt}
IDAInitB allocates and initializes backward memory for CVODES.
\end{alltt}

\end{samepage}


%% definition  -------------------

\begin{samepage}

\noindent{\bf \sc Synopsis}

\begin{alltt}
function idxB = IDAInitB(fctB, tB0, yyB0, ypB0, optionsB) 
\end{alltt}

\end{samepage}

%% description -------------------

\noindent{\bf \sc Description}

\begin{alltt}
IDAInitB allocates and initializes backward memory for CVODES.

   Usage:   IDXB = IDAInitB ( DAEFUNB, TB0, YYB0, YPB0 [, OPTIONSB] )

   DAEFUNB  is a function defining the adjoint DAE: F(t,y,y',yB,yB')=0
            This function must return a vector containing the current 
            value of the adjoint DAE residual.
   TB0      is the final value of t.
   YYB0     is the final condition vector yB(tB0).  
   YPB0     is the final condition vector yB'(tB0).  
   OPTIONSB is an (optional) set of integration options, created with
            the IDASetOptions function. 

   IDAInitB returns the index IDXB associated with this backward
   problem. This index must be passed as an argument to any subsequent
   functions related to this backward problem.

   See also: IDASetOptions, IDAResFnB
\end{alltt}






\vspace{0.1in}