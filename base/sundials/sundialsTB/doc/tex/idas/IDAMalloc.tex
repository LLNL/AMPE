%% foo.tex
\begin{samepage}
\hrule
\begin{center}
\phantomsection
{\large \verb!IDAMalloc!}
\label{p:IDAMalloc}
\index{IDAMalloc}
\end{center}
\hrule\vspace{0.1in}

%% one line -------------------

\noindent{\bf \sc Purpose}

\begin{alltt}
IDAMalloc allocates and initializes memory for IDAS.
\end{alltt}

\end{samepage}


%% definition  -------------------

\begin{samepage}

\noindent{\bf \sc Synopsis}

\begin{alltt}
function [] = IDAMalloc(fct,t0,yy0,yp0,varargin) 
\end{alltt}

\end{samepage}

%% description -------------------

\noindent{\bf \sc Description}

\begin{alltt}
IDAMalloc allocates and initializes memory for IDAS.

   Usage: IDAMalloc ( DAEFUN, T0, YY0, YP0 [, OPTIONS [, DATA] ] ) 

   DAEFUN   is a function defining the DAE residual: f(t,yy,yp).
            This function must return a vector containing the current 
            value of the residual.
   T0       is the initial value of t.
   YY0      is the initial condition vector y(t0).  
   YP0      is the initial condition vector y'(t0).  
   OPTIONS  is an (optional) set of integration options, created with
            the IDASetOptions function. 
   DATA     is (optional) problem data passed unmodified to all
            user-provided functions when they are called. For example,
            YD = DAEFUN(T,YY,YP,DATA).

  See also: IDARhsFn
\end{alltt}






\vspace{0.1in}