%% foo.tex
\begin{samepage}
\hrule
\begin{center}
\phantomsection
{\large \verb!IDAReInitB!}
\label{p:IDAReInitB}
\index{IDAReInitB}
\end{center}
\hrule\vspace{0.1in}

%% one line -------------------

\noindent{\bf \sc Purpose}

\begin{alltt}
IDAReInitB allocates and initializes backward memory for IDAS.
\end{alltt}

\end{samepage}


%% definition  -------------------

\begin{samepage}

\noindent{\bf \sc Synopsis}

\begin{alltt}
function [] = IDAReInitB(idxB,tB0,yyB0,ypB0,optionsB) 
\end{alltt}

\end{samepage}

%% description -------------------

\noindent{\bf \sc Description}

\begin{alltt}
IDAReInitB allocates and initializes backward memory for IDAS.
   where a prior call to IDAInitB has been made with the same
   problem size NB. IDAReInitB performs the same input checking
   and initializations that IDAInitB does, but it does no 
   memory allocation, assuming that the existing internal memory 
   is sufficient for the new problem.

   Usage:   IDAReInitB ( IDXB, TB0, YYB0, YPB0 [, OPTIONSB] )

   IDXB     is the index of the backward problem, returned by
            IDAInitB.
   TB0      is the final value of t.
   YYB0     is the final condition vector yB(tB0).  
   YPB0     is the final condition vector yB'(tB0).
   OPTIONSB is an (optional) set of integration options, created with
            the IDASetOptions function. 

   See also: IDASetOptions, IDAInitB
\end{alltt}






\vspace{0.1in}