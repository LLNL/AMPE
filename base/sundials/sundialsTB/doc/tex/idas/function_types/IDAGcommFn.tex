%% # foo.tex
\begin{samepage}
\hrule
\begin{center}
\phantomsection
{\large \verb!IDAGcommFn!}
\label{p:IDAGcommFn}
\index{IDAGcommFn}
\end{center}
\hrule\vspace{0.1in}

%% one line -------------------

\noindent{\bf \sc Purpose}

\begin{alltt}
IDAGcommFn - type for communication function (BBDPre).
\end{alltt}

\end{samepage}


%% definition  -------------------

\begin{samepage}

\noindent{\bf \sc Synopsis}

\begin{alltt}
This is a script file. 
\end{alltt}

\end{samepage}

%% description -------------------

\noindent{\bf \sc Description}

\begin{alltt}
IDAGcommFn - type for communication function (BBDPre).

   The function GCOMFUN must be defined as 
        FUNCTION FLAG = GCOMFUN(T, YY, YP)
   and can be used to perform all interprocess communication necessary
   to evaluate the approximate residual function for the BBDPre
   preconditioner module.
   If a user data structure DATA was specified in IDAMalloc, then
   GCOMFUN must be defined as
        FUNCTION [FLAG, NEW_DATA] = GCOMFUN(T, YY, YP, DATA)
   If the local modifications to the user data structure are needed 
   in other user-provided functions then the GCOMFUN function must also 
   set NEW_DATA. Otherwise, it should set NEW_DATA=[] (do not set 
   NEW_DATA = DATA as it would lead to unnecessary copying).

   The function GCOMFUN must set FLAG=0 if successful, FLAG&lt;0 if an
   unrecoverable failure occurred, or FLAG&gt;0 if a recoverable error
   occurred.

   See also IDAGlocalFn, IDASetOptions

   NOTES:
     GCOMFUN is specified through the GcommFn property in IDASetOptions
     and is used only if the property PrecModule is set to 'BBDPre'.

     Each call to GCOMFUN is preceded by a call to the residual function
     DAEFUN with the same arguments T, YY, and YP.
     Thus GCOMFUN can omit any communication done by DAEFUN if relevant
     to the evaluation of G by GLOCFUN. If all necessary communication
     was done by DAEFUN, GCOMFUN need not be provided.
\end{alltt}






\vspace{0.1in}