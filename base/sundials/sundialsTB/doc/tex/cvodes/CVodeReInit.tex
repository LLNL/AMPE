%% foo.tex
\begin{samepage}
\hrule
\begin{center}
\phantomsection
{\large \verb!CVodeReInit!}
\label{p:CVodeReInit}
\index{CVodeReInit}
\end{center}
\hrule\vspace{0.1in}

%% one line -------------------

\noindent{\bf \sc Purpose}

\begin{alltt}
CVodeReInit reinitializes memory for CVODES
\end{alltt}

\end{samepage}


%% definition  -------------------

\begin{samepage}

\noindent{\bf \sc Synopsis}

\begin{alltt}
function CVodeReInit(t0, y0, options) 
\end{alltt}

\end{samepage}

%% description -------------------

\noindent{\bf \sc Description}

\begin{alltt}
CVodeReInit reinitializes memory for CVODES
   where a prior call to CVodeInit has been made with the same
   problem size N. CVodeReInit performs the same input checking
   and initializations that CVodeInit does, but it does no 
   memory allocation, assuming that the existing internal memory 
   is sufficient for the new problem.

   Usage: CVodeReInit ( T0, Y0 [, OPTIONS ] ) 

   T0       is the initial value of t.
   Y0       is the initial condition vector y(t0).  
   OPTIONS  is an (optional) set of integration options, created with
            the CVodeSetOptions function. 

   See also: CVodeSetOptions, CVodeInit
\end{alltt}






\vspace{0.1in}