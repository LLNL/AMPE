%% foo.tex
\begin{samepage}
\hrule
\begin{center}
\phantomsection
{\large \verb!CVodeQuadReInitB!}
\label{p:CVodeQuadReInitB}
\index{CVodeQuadReInitB}
\end{center}
\hrule\vspace{0.1in}

%% one line -------------------

\noindent{\bf \sc Purpose}

\begin{alltt}
CVodeQuadReInitB reinitializes memory for backward quadrature integration.
\end{alltt}

\end{samepage}


%% definition  -------------------

\begin{samepage}

\noindent{\bf \sc Synopsis}

\begin{alltt}
function [] = CVodeQuadReInitB(idxB, yQB0, optionsB) 
\end{alltt}

\end{samepage}

%% description -------------------

\noindent{\bf \sc Description}

\begin{alltt}
CVodeQuadReInitB reinitializes memory for backward quadrature integration.

   Usage: CVodeQuadReInitB ( IDXB, YS0 [, OPTIONS ] ) 

   IDXB     is the index of the backward problem, returned by
            CVodeInitB.
   YQB0     is the final conditions vector yQB(tB0).
   OPTIONS  is an (optional) set of QUAD options, created with
            the CVodeSetQuadOptions function. 

   See also: CVodeSetQuadOptions, CVodeReInitB, CVodeQuadInitB
\end{alltt}






\vspace{0.1in}