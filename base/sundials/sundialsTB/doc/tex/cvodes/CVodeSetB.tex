%% foo.tex
\begin{samepage}
\hrule
\begin{center}
\phantomsection
{\large \verb!CVodeSetB!}
\label{p:CVodeSetB}
\index{CVodeSetB}
\end{center}
\hrule\vspace{0.1in}

%% one line -------------------

\noindent{\bf \sc Purpose}

\begin{alltt}
CVodeSetB changes optional input values during the integration.
\end{alltt}

\end{samepage}


%% definition  -------------------

\begin{samepage}

\noindent{\bf \sc Synopsis}

\begin{alltt}
function CVodeSetB(idxB, varargin) 
\end{alltt}

\end{samepage}

%% description -------------------

\noindent{\bf \sc Description}

\begin{alltt}
CVodeSetB changes optional input values during the integration.

   Usage: CVodeSetB( IDXB, 'NAME1',VALUE1,'NAME2',VALUE2,... )

   CVodeSetB can be used to change some of the optional inputs for
   the backward problem identified by IDXB during the backward
   integration, i.e., without need for a solver reinitialization.
   The property names accepted by CVodeSet are a subset of those valid
   for CVodeSetOptions. Any unspecified properties are left unchanged.
   
   CVodeSetB with no input arguments displays all property names.

CVodeSetB properties
(See also the CVODES User Guide)

UserData - problem data passed unmodified to all user functions.
  Set VALUE to be the new user data.
RelTol - Relative tolerance
  Set VALUE to the new relative tolerance
AbsTol - absolute tolerance
  Set VALUE to be either the new scalar absolute tolerance or
  a vector of absolute tolerances, one for each solution component.
\end{alltt}






\vspace{0.1in}