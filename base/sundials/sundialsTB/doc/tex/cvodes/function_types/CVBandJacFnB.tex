%% # foo.tex
\begin{samepage}
\hrule
\begin{center}
\phantomsection
{\large \verb!CVBandJacFnB!}
\label{p:CVBandJacFnB}
\index{CVBandJacFnB}
\end{center}
\hrule\vspace{0.1in}

%% one line -------------------

\noindent{\bf \sc Purpose}

\begin{alltt}
CVBandJacFnB - type for user provided banded Jacobian function for backward problems.
\end{alltt}

\end{samepage}


%% definition  -------------------

\begin{samepage}

\noindent{\bf \sc Synopsis}

\begin{alltt}
This is a script file. 
\end{alltt}

\end{samepage}

%% description -------------------

\noindent{\bf \sc Description}

\begin{alltt}
CVBandJacFnB - type for user provided banded Jacobian function for backward problems.

   The function BJACFUNB must be defined either as
        FUNCTION [JB, FLAG] = BJACFUNB(T, Y, YB, FYB)
   or as
        FUNCTION [JB, FLAG, NEW_DATA] = BJACFUNB(T, Y, YB, FYB, DATA)
   depending on whether a user data structure DATA was specified in
   CVodeMalloc. In either case, it must return the matrix JB, the
   Jacobian of fB(t,y,yB), with respect to yB. The input argument
   FYB contains the current value of f(t,y,yB).

   The function BJACFUNB must set FLAG=0 if successful, FLAG&lt;0 if an
   unrecoverable failure occurred, or FLAG&gt;0 if a recoverable error
   occurred.

   See also CVodeSetOptions

   See the CVODES user guide for more informaiton on the structure of
   a banded Jacobian.

   NOTE: BJACFUNB is specified through the property JacobianFn to
   CVodeSetOptions and is used only if the property LinearSolver
   was set to 'Band'.
\end{alltt}






\vspace{0.1in}