%% # foo.tex
\begin{samepage}
\hrule
\begin{center}
\phantomsection
{\large \verb!CVJacTimesVecFn!}
\label{p:CVJacTimesVecFn}
\index{CVJacTimesVecFn}
\end{center}
\hrule\vspace{0.1in}

%% one line -------------------

\noindent{\bf \sc Purpose}

\begin{alltt}
CVJacTimesVecFn - type for user provided Jacobian times vector function.
\end{alltt}

\end{samepage}


%% definition  -------------------

\begin{samepage}

\noindent{\bf \sc Synopsis}

\begin{alltt}
This is a script file. 
\end{alltt}

\end{samepage}

%% description -------------------

\noindent{\bf \sc Description}

\begin{alltt}
CVJacTimesVecFn - type for user provided Jacobian times vector function.

   The function JTVFUN must be defined as 
        FUNCTION [JV, FLAG] = JTVFUN(T,Y,FY,V)
   and must return a vector JV corresponding to the product of the 
   Jacobian of f(t,y) with the vector v.
   The input argument FY contains the current value of f(t,y).
   If a user data structure DATA was specified in CVodeMalloc, then
   JTVFUN must be defined as
        FUNCTION [JV, FLAG, NEW_DATA] = JTVFUN(T,Y,FY,V,DATA)
   If the local modifications to the user data structure are needed in
   other user-provided functions then, besides setting the vector JV,
   the JTVFUN function must also set NEW_DATA. Otherwise, it should set
   NEW_DATA=[] (do not set NEW_DATA = DATA as it would lead to
   unnecessary copying).

   The function JTVFUN must set FLAG=0 if successful, or FLAG~=0 if
   a failure occurred.

   See also CVodeSetOptions

   NOTE: JTVFUN is specified through the property JacobianFn to
   CVodeSetOptions and is used only if the property LinearSolver
   was set to 'GMRES', 'BiCGStab', or 'TFQMR'.
\end{alltt}






\vspace{0.1in}