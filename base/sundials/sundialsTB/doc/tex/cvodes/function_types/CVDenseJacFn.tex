%% # foo.tex
\begin{samepage}
\hrule
\begin{center}
\phantomsection
{\large \verb!CVDenseJacFn!}
\label{p:CVDenseJacFn}
\index{CVDenseJacFn}
\end{center}
\hrule\vspace{0.1in}

%% one line -------------------

\noindent{\bf \sc Purpose}

\begin{alltt}
CVDenseJacFn - type for user provided dense Jacobian function.
\end{alltt}

\end{samepage}


%% definition  -------------------

\begin{samepage}

\noindent{\bf \sc Synopsis}

\begin{alltt}
This is a script file. 
\end{alltt}

\end{samepage}

%% description -------------------

\noindent{\bf \sc Description}

\begin{alltt}
CVDenseJacFn - type for user provided dense Jacobian function.

   The function DJACFUN must be defined as 
        FUNCTION [J, FLAG] = DJACFUN(T, Y, FY)
   and must return a matrix J corresponding to the Jacobian of f(t,y).
   The input argument FY contains the current value of f(t,y).
   If a user data structure DATA was specified in CVodeMalloc, then
   DJACFUN must be defined as
        FUNCTION [J, FLAG, NEW_DATA] = DJACFUN(T, Y, FY, DATA)
   If the local modifications to the user data structure are needed in
   other user-provided functions then, besides setting the matrix J,
   the DJACFUN function must also set NEW_DATA. Otherwise, it should
   set NEW_DATA=[] (do not set NEW_DATA = DATA as it would lead to 
   unnecessary copying).

   The function DJACFUN must set FLAG=0 if successful, FLAG&lt;0 if an
   unrecoverable failure occurred, or FLAG&gt;0 if a recoverable error
   occurred.
   See also CVodeSetOptions

   NOTE: DJACFUN is specified through the property JacobianFn to
   CVodeSetOptions and is used only if the property LinearSolver
   was set to 'Dense'.
\end{alltt}






\vspace{0.1in}