%% foo.tex
\begin{samepage}
\hrule
\begin{center}
\phantomsection
{\large \verb!CVodeSensInit!}
\label{p:CVodeSensInit}
\index{CVodeSensInit}
\end{center}
\hrule\vspace{0.1in}

%% one line -------------------

\noindent{\bf \sc Purpose}

\begin{alltt}
CVodeSensInit allocates and initializes memory for FSA with CVODES.
\end{alltt}

\end{samepage}


%% definition  -------------------

\begin{samepage}

\noindent{\bf \sc Synopsis}

\begin{alltt}
function [] = CVodeSensInit(Ns,fctS,yS0,options) 
\end{alltt}

\end{samepage}

%% description -------------------

\noindent{\bf \sc Description}

\begin{alltt}
CVodeSensInit allocates and initializes memory for FSA with CVODES.

   Usage: CVodeSensInit ( NS, SFUN, YS0 [, OPTIONS ] ) 

   NS       is the number of parameters with respect to which sensitivities
            are desired
   SFUN     is a function defining the righ-hand sides of the sensitivity
            ODEs yS' = fS(t,y,yS).
   YS0      Initial conditions for sensitivity variables.
            YS0 must be a matrix with N rows and Ns columns, where N is the problem
            dimension and Ns the number of sensitivity systems.
   OPTIONS  is an (optional) set of FSA options, created with
            the CVodeSetFSAOptions function. 

   See also CVodeSensSetOptions, CVodeInit, CVSensRhsFn
\end{alltt}






\vspace{0.1in}