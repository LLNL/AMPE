%% foo.tex
\begin{samepage}
\hrule
\begin{center}
\phantomsection
{\large \verb!CVodeMonitorB!}
\label{p:CVodeMonitorB}
\index{CVodeMonitorB}
\end{center}
\hrule\vspace{0.1in}

%% one line -------------------

\noindent{\bf \sc Purpose}

\begin{alltt}
CVodeMonitorB is the default CVODES monitoring function for backward problems.
\end{alltt}

\end{samepage}


%% definition  -------------------

\begin{samepage}

\noindent{\bf \sc Synopsis}

\begin{alltt}
function [new_data] = CVodeMonitorB(call, idxB, T, Y, YQ, data) 
\end{alltt}

\end{samepage}

%% description -------------------

\noindent{\bf \sc Description}

\begin{alltt}
CVodeMonitorB is the default CVODES monitoring function for backward problems.
   To use it, set the Monitor property in CVodeSetOptions to
   'CVodeMonitorB' or to @CVodeMonitorB and 'MonitorData' to mondata
   (defined as a structure).
  
   With default settings, this function plots the evolution of the step 
   size, method order, and various counters.
   
   Various properties can be changed from their default values by passing
   to CVodeSetOptions, through the property 'MonitorData', a structure
   MONDATA with any of the following fields. If a field is not defined, 
   the corresponding default value is used.

   Fields in MONDATA structure:
     o stats [ {true} | false ]
         If true, report the evolution of the step size and method order.
     o cntr [ {true} | false ]
         If true, report the evolution of the following counters:
         nst, nfe, nni, netf, ncfn (see CVodeGetStats)
     o mode [ {'graphical'} | 'text' | 'both' ] 
         In graphical mode, plot the evolutions of the above quantities.
         In text mode, print a table.
     o sol  [ true | {false} ]
         If true, plot solution components.
     o select [ array of integers ]
         To plot only particular solution components, specify their indeces in
         the field select. If not defined, but sol=true, all components are plotted.
     o updt [ integer | {50} ]
         Update frequency. Data is posted in blocks of dimension n.
     o skip [ integer | {0} ]
         Number of integrations steps to skip in collecting data to post.
     o post [ {true} | false ]
         If false, disable all posting. This option is necessary to disable
         monitoring on some processors when running in parallel.

   See also CVodeSetOptions, CVMonitorFnB

   NOTES:
     1. The argument mondata is REQUIRED. Even if only the default options
        are desired, set mondata=struct; and pass it to CVodeSetOptions.
     2. The yQ argument is currently ignored.
\end{alltt}






\vspace{0.1in}