%% # foo.tex
\begin{samepage}
\hrule
\begin{center}
\phantomsection
{\large \verb!KINDenseJacFn!}
\label{p:KINDenseJacFn}
\index{KINDenseJacFn}
\end{center}
\hrule\vspace{0.1in}

%% one line -------------------

\noindent{\bf \sc Purpose}

\begin{alltt}
KINDenseJacFn - type for user provided dense Jacobian function.
\end{alltt}

\end{samepage}


%% definition  -------------------

\begin{samepage}

\noindent{\bf \sc Synopsis}

\begin{alltt}
This is a script file. 
\end{alltt}

\end{samepage}

%% description -------------------

\noindent{\bf \sc Description}

\begin{alltt}
KINDenseJacFn - type for user provided dense Jacobian function.

   The function DJACFUN must be defined as 
        FUNCTION [J, FLAG] = DJACFUN(Y,FY)
   and must return a matrix J corresponding to the Jacobian of f(y).
   The input argument FY contains the current value of f(y).
   If a user data structure DATA was specified in KINMalloc, then
   DJACFUN must be defined as
        FUNCTION [J, FLAG, NEW_DATA] = DJACFUN(Y,FY,DATA)
   If the local modifications to the user data structure are needed in
   other user-provided functions then, besides setting the matrix J and
   the flag FLAG, the DJACFUN function must also set NEW_DATA. Otherwise, 
   it should set NEW_DATA=[] (do not set NEW_DATA = DATA as it would lead
   to unnecessary copying).

   The function DJACFUN must set FLAG=0 if successful, FLAG&lt;0 if an
   unrecoverable failure occurred, or FLAG&gt;0 if a recoverable error
   occurred.

   See also KINSetOptions

   NOTE: DJACFUN is specified through the property JacobianFn to KINSetOptions
   and is used only if the property LinearSolver was set to 'Dense'.
\end{alltt}






\vspace{0.1in}