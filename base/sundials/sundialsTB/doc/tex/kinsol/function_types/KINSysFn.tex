%% # foo.tex
\begin{samepage}
\hrule
\begin{center}
\phantomsection
{\large \verb!KINSysFn!}
\label{p:KINSysFn}
\index{KINSysFn}
\end{center}
\hrule\vspace{0.1in}

%% one line -------------------

\noindent{\bf \sc Purpose}

\begin{alltt}
KINSysFn - type for user provided system function
\end{alltt}

\end{samepage}


%% definition  -------------------

\begin{samepage}

\noindent{\bf \sc Synopsis}

\begin{alltt}
This is a script file. 
\end{alltt}

\end{samepage}

%% description -------------------

\noindent{\bf \sc Description}

\begin{alltt}
KINSysFn - type for user provided system function

   The function SYSFUN must be defined as 
        FUNCTION [FY, FLAG] = SYSFUN(Y)
   and must return a vector FY corresponding to f(y).
   If a user data structure DATA was specified in KINMalloc, then
   SYSFUN must be defined as
        FUNCTION [FY, FLAG, NEW_DATA] = SYSFUN(Y,DATA)
   If the local modifications to the user data structure are needed 
   in other user-provided functions then, besides setting the vector FY,
   the SYSFUN function must also set NEW_DATA. Otherwise, it should set
   NEW_DATA=[] (do not set NEW_DATA = DATA as it would lead to
   unnecessary copying).

   The function SYSFUN must set FLAG=0 if successful, FLAG&lt;0 if an
   unrecoverable failure occurred, or FLAG&gt;0 if a recoverable error
   occurred.

   See also KINMalloc

   NOTE: SYSFUN is specified through the KINMalloc function.
\end{alltt}






\vspace{0.1in}