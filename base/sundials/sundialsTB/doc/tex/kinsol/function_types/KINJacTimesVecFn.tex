%% # foo.tex
\begin{samepage}
\hrule
\begin{center}
\phantomsection
{\large \verb!KINJacTimesVecFn!}
\label{p:KINJacTimesVecFn}
\index{KINJacTimesVecFn}
\end{center}
\hrule\vspace{0.1in}

%% one line -------------------

\noindent{\bf \sc Purpose}

\begin{alltt}
KINJacTimesVecFn - type for user provided Jacobian times vector function.
\end{alltt}

\end{samepage}


%% definition  -------------------

\begin{samepage}

\noindent{\bf \sc Synopsis}

\begin{alltt}
This is a script file. 
\end{alltt}

\end{samepage}

%% description -------------------

\noindent{\bf \sc Description}

\begin{alltt}
KINJacTimesVecFn - type for user provided Jacobian times vector function.

   The function JTVFUN must be defined as 
        FUNCTION [JV, NEW_Y, FLAG] = JTVFUN(Y, V, NEW_Y)
   and must return a vector JV corresponding to the product of the 
   Jacobian of f(y) with the vector v. On input, NEW_Y indicates if
   the iterate has been updated in the interim. JV must be update
   or reevaluated, if appropriate, unless NEW_Y=false. This flag must
   be reset by the user.
   If a user data structure DATA was specified in KINMalloc, then
   JTVFUN must be defined as
        FUNCTION [JV, NEW_Y, FLAG, NEW_DATA] = JTVFUN(Y, V, NEW_Y, DATA)
   If the local modifications to the user data structure are needed in
   other user-provided functions then, besides setting the vector JV, and
   flags NEW_Y and FLAG, the JTVFUN function must also set NEW_DATA. Otherwise, 
   it should set NEW_DATA=[] (do not set NEW_DATA = DATA as it would lead to
   unnecessary copying).

   If successful, FLAG should be set to 0. If an error occurs, FLAG should
   be set to a nonzero value.

   See also KINSetOptions

   NOTE: JTVFUN is specified through the property JacobianFn to KINSetOptions
   and is used only if the property LinearSolver was set to 'GMRES' or 'BiCGStab'.
\end{alltt}






\vspace{0.1in}