\chapter{Further model derivation}
\label{sec:appendix}

\section{Functional derivatives}
\label{sec:fderiv}

Let
%
\begin{equation}
  F(u) = F(u_1, \ldots, u_n) = \int_{\Omega} I(u_1, \ldots, u_n) d\Omega
\end{equation}
%
be a functional on a space of function tuples $u = (u_1,\ldots,u_n)$
where the $u_i$ are defined and periodic on the domain
$\Omega$.  By definition, the functional derivative of $F$ with
respect to $u_i$ at any point $\hat{u} = (\hat{u}_1,\ldots,\hat{u}_n)$
is the functional whose action is defined by
%
\begin{equation}
  \left .  \left < \frac{\delta F}{\delta
      u_i}(\hat{u}), v \right >
  = \frac{d}{d \epsilon} \int_{\Omega} I(\hat{u}_1, \ldots, \hat{u}_i + \epsilon
  v, \ldots, \hat{u}_n) d\Omega \right |_{\epsilon=0} ~~{\rm for~all~}v.
\end{equation}
%
Hence, for all $v$,
%
\begin{eqnarray}
  \left < \frac{\delta F}{\delta
      u_i}(\hat{u}), v \right >
  &=& \int_{\Omega} \left ( \frac{\partial I}{\partial u_i}(\hat{u})v + \frac{\partial
    I}{\partial \nabla u_i}(\hat{u}) \cdot \nabla v \right ) d\Omega
      \\
  &=& \int_{\Omega} \left [ \frac{\partial I}{\partial u_i}(\hat{u})v +
      \nabla \cdot \left ( \frac{\partial
    I}{\partial u_i}(\hat{u}) v \right ) 
  - \left ( \nabla \cdot \frac{\partial
    I}{\partial u_i}(\hat{u}) \right ) v
  \right ] d\Omega \label{eq:divterm} \\
  &=& \left < \frac{\partial I}{\partial u_i}(\hat{u})
  - \nabla \cdot \frac{\partial
    I}{\partial u_i}(\hat{u}) , v
  \right > ,
\end{eqnarray}
%
where the second term in (\ref{eq:divterm}) vanishes due to the
divergence theorem and the assumed periodicity of $v$.
Thus,
%
\begin{equation}
  \frac{\delta F}{\delta u_i} = \frac{\partial I}{\partial u_i}
  - \nabla \cdot \frac{\partial
    I}{\partial \nabla u_i} .  \label{funcderiv}
\end{equation}

%%%%%%%%%%%%%%%%%%%%%%%%%%%%%%%%%%%%%%%%%%%%%%%%%%%%%%%%%%%%%%%%%%%%%%%%%%%%%%%%%
%
\section{Quaternions}
\label{sec:quaternions}

By analogy with complex numbers, a quaternion $q$ can be written as a
linear combination
%
\begin{equation}
  q = (a+ib+jc+kd)
\end{equation}
%
with 3 imaginary dimensions $i,j,k$.  In the following we will assume
that all of the quaternions are normalized, {\it i.e.}
$a^2+b^2+c^2+d^2=1$.  A rotation by an angle $\theta$ around an axis
of direction $(x,y,z)$ can be described by a quaternion
%
\begin{equation}
  q=\cos(\theta/2)+i(x*\sin(\theta/2))+j(y*\sin(\theta/2))+k(z*\sin(\theta/2))
\end{equation}
%
where $(x,y,z)$ is assumed to be normalized.  We denote by
%
\begin{equation}
  q^*=a-ib-jc-kd
\end{equation}
%
the conjugate of $q$.  Using the quaternion multiplication
rules, {\em i.e.}, $ij = k, jk = i, ki = j, ji = -k, kj = -i, ik = -j, i^2 =
j^2 = k^2 = -1$, the rotation between two quaternions, $q_1$ and
$q_2$, can be computed as either $q_1^* q_2$ or $q_2 q_1^*$, depending
on an arbitrary convention.  Due to the non-commutative nature of the
quaternion multiplication, these two are not equivalent, but are the
conjugate of each other.  They correspond to a rotation by the same
angle around an axis of opposite direction.

The formula for the misorientation angle between two unit-length
quaternions is given by
%
\begin{equation}
  |d_{21}| = 2\sin(\theta_{12}/4).
\label{eq:angle}
\end{equation}
%
where $|d_{21}|$ is the distance between $q_1$ and $q_2$ in the 4D
space, and $\theta_{12}$ is the angle between $q_1$ and $q_2$ (note
the factor of 2).  The mapping from unit quaternions to rotations is
2-to-1.  Multiplying a quaternion by an overall factor of -1 has no
physical effect.  Two quaternions on opposite sides of the hypersphere
are a distance $|d_{21}| = 2$ apart, which gives $\theta_{12} = 2\pi$,
which is no rotation at all.  To first order, (\ref{eq:angle}) gives
%
\begin{equation}
   \theta\simeq 2|d_{21}|
\end{equation}
%
and justifies using quaternion differences to approximate local
misorientation \cite{0295-5075-71-1-131}.

%%%%%%%%%%%%%%%%%%%%%%%%%%%%%%%%%%%%%%%%%%%%%%%%%%%%%%%%%%%%%%%%%%%%%%%%%%%%%%%%%
%
\section{Derivation of the Kim, Kim, Suzuki (KKS) model}
\label{sec:kks}

In \cite{PhysRevE.60.7186}, Kim {\em et al.}
define the relation between the variables $(c_S,c_L)$, and
$(c,\phi)$ in (\ref{eq:cmix}) by imposing the condition
%
\begin{equation}
  \left.\frac{\partial f^S}{\partial c_S}\right|_{c_S=c_S(x,t)} =
  \left.\frac{\partial f^L}{\partial c_L}\right|_{c_L=c_L(x,t)} =
  \mu(x,t).
\label{eq:mu}
\end{equation}
%
This means that the chemical potential is equal for both phases
at the infinitesimal point $x$, and thus there would be no change
in free energy by exchanging an infinitesimal amount of species
between phases S and L.

In the KKS model, we also define
%
\begin{equation}
  D_c(c,\phi,T)=D_c^0(\phi,T)\left(\frac{\partial^2 f}{\partial
  c^2}\right)^{-1}.
\label{eq:diffcoeff}
\end{equation}
%
From (\ref{eq:fmix}) and (\ref{eq:mu}), we obtain
%
\begin{eqnarray}
  \frac{\partial f}{\partial c} &=& h(\phi)\frac{\partial
  f^S}{\partial c_S}\frac{\partial c_S}{\partial c} +
  [1-h(\phi)]\frac{\partial f^L}{\partial c_L}\frac{\partial
  c_L}{\partial c} \nonumber \\ &=&
  \mu\left(h(\phi)\frac{\partial c_S}{\partial c} +
  [1-h(\phi)]\frac{\partial c_L}{\partial c}\right)
\label{eq:dfdc} 
\end{eqnarray}
%
and from (\ref{eq:cmix})
%
\begin{equation}
  1 = \frac{\partial c}{\partial c} = h(\phi)\frac{\partial
  c_S}{\partial c} + [1-h(\phi)]\frac{\partial c_L}{\partial c}
\label{eq:dcdc}
\end{equation}
%
From (\ref{eq:dfdc}) and (\ref{eq:dcdc}), we then get
%
\begin{equation}
  \left.\frac{\partial f}{\partial c}\right|_{c=c(x,t)} = \mu(x,t)
\label{eq:dfdcmu}
\end{equation}
%
(Eq.~28 of \cite{PhysRevE.60.7186}).

From (\ref{eq:fmix}), we also have
%
\begin{eqnarray}
  \frac{\partial f}{\partial\phi} = & & h'(\phi)(f^S(c_S)-f^L(c_L))+
  \omega g'(\phi)
  \nonumber \\
  & & {} + h(\phi)\mu\frac{\partial c_S}{\partial\phi} +
  (1-h(\phi))\mu\frac{\partial c_L}{\partial\phi}.
\label{eq:dfdphi}
\end{eqnarray}
%
Noticing that, since $c$ is independent of $\phi$,
%
\begin{eqnarray}
  h(\phi)\frac{\partial c_S}{\partial\phi} +
  (1-h(\phi))\frac{\partial c_L}{\partial\phi} &=& \frac{\partial
  c}{\partial\phi} - h'(\phi)(c_S-c_L) \nonumber \\
  &=& - h'(\phi)(c_S-c_L),
\end{eqnarray}
%
we get
%
\begin{equation}
  \frac{\partial f}{\partial\phi} =
  - h'(\phi) \left( f^L(c_L) - f^S(c_S) - \mu (c_L - c_S) \right)
  + \omega g'(\phi)
\label{eq:dfdphi2}
\end{equation}
%
(Eq. (27) of \cite{PhysRevE.60.7186}). We also get
%
\begin{eqnarray}
  \frac{\partial}{\partial c}\left(\frac{\partial
  f}{\partial\phi}\right) &=& -h'(\phi) \left[\frac{\partial
  f^L}{\partial c_L}\frac{\partial c_L}{\partial c} - \frac{\partial
  f^S}{\partial c_S}\frac{\partial c_S}{\partial c}
  -\mu\left(\frac{\partial c_L}{\partial c}-\frac{\partial
  c_S}{\partial c}\right) - \frac{\partial^2 f}{\partial
  c^2}(c_L-c_S)\right] \nonumber \\ &=&
  h'(\phi)(c_L-c_S)\frac{\partial^2 f}{\partial c^2}
\label{eq:dfdcdphi}
\end{eqnarray}
%
using (\ref{eq:mu}). This is Eq. (30) of
\cite{PhysRevE.60.7186}.

Now, since $f$ is a function of $c$ and $\phi$,
%
\begin{equation}
  \frac{\partial}{\partial x}\frac{\partial f}{\partial c} =
  \frac{\partial^2 f}{\partial c^2}\frac{\partial c}{\partial x} +
  \frac{\partial^2 f}{\partial c
  \partial\phi}\frac{\partial\phi}{\partial x}
\end{equation}
%
and we have
%
\begin{equation}
  \nabla\mu = \nabla\frac{\partial f}{\partial c} = \frac{\partial^2
  f}{\partial c^2}\nabla c + \frac{\partial^2 f}{\partial
  c\partial\phi}\nabla\phi
\label{eq:gradmu}
\end{equation}
%
From Eqs. (\ref{eq:crelax}),
(\ref{eq:diffcoeff}),(\ref{eq:dfdcdphi}),(\ref{eq:gradmu}), we
obtain
%
\begin{equation}
  \frac{\partial c}{\partial t}
  =M_c\nabla\cdot D_c^0(\phi,T)\nabla c+ M_c\nabla\cdot
  D_c^0(\phi,T)h'(\phi)(c_L-c_S)\nabla\phi
\label{eq:ct}
\end{equation}
%
which is Eq. (33) of \cite{PhysRevE.60.7186}.

To actually compute the right hand side of (\ref{eq:ct}), we need to
know $c_S(c,\phi)$ and $c_L(c,\phi)$. For that, we need to know an
explicit form of $f^S$ and $f^L$. See, for example, Section
\ref{sec:hbsm}.

%%%%%%%%%%%%%%%%%%%%%%%%%%%%%%%%%%%%%%%%%%%%%%%%%%%%%%%%%%%%%%%%%%%%%%%%%%%%%%%%%
%
\section{Hu, Baskes, Stan, Mitchell (HBSM) model for a binary alloy}
\label{sec:hbsm}

In \cite{HuBaskesStanMitchell07}, Hu {\em et al.} propose a phase-field
model for a binary alloy.  The two phases are $\epsilon$, or body-centered cubic (bcc) and $\delta$, or face-centered cubic (fcc) (to
substitute for L and S, respectively, in the preceding section).  The following explicit
forms for $f^\epsilon$ and $f^\delta$ are proposed:
%
\begin{equation}
  f^\epsilon(c_\epsilon,T)=A_\epsilon\left(c_\epsilon-c_\epsilon^{eq}(T)\right)^2,
\label{eq:fepsilon}
\end{equation}
%
\begin{equation}
  f^\delta(c_\delta,T)=A_\delta\left(c_\delta-c_\delta^{eq}(T)\right)^2.
\label{eq:fdelta}
\end{equation}
%
Using (\ref{eq:mu}), we obtain
%
\begin{equation}
  A_\epsilon(c_\epsilon-c_\epsilon^{eq})=A_\delta(c_\delta-c_\delta^{eq})
\end{equation}
%
and thus
%
\begin{equation}
  c_\epsilon(c_\delta,T) =
  c_\epsilon^{eq}+A_\delta(c_\delta-c_\delta^{eq}(T))/A_\epsilon
\end{equation}
%
\begin{equation}
  c_\delta(c_\epsilon,T) =
  c_\delta^{eq}+A_\epsilon(c_\epsilon-c_\epsilon^{eq}(T))/A_\delta
\end{equation}
%
From (\ref{eq:cmix}), we then get
%
\begin{equation}
  c_\epsilon(c,\phi,T) =
  \frac{c-h(\phi)\left(c_\delta^{eq}(T)-\frac{A_\epsilon}{A_\delta}c_\epsilon^{eq}(T)\right)}
  {(1-h(\phi))+h(\phi)\frac{A_\epsilon}{A_\delta}}
\label{eq:cepsilon}
\end{equation}
%
\begin{equation}
  c_\delta(c,\phi,T) =
  \frac{c-(1-h(\phi))\left(c_\epsilon^{eq}(T)-\frac{A_\delta}{A_\epsilon}c_\delta^{eq}(T)\right)}
  {h(\phi)+(1-h(\phi))\frac{A_\delta}{A_\epsilon}}
\label{eq:cdelta}
\end{equation}
%
These expressions for $c_\epsilon(c,\phi,T)$ and $c_\delta(c,\phi,T)$
can be substituted into (\ref{eq:ct}) to have a right-hand side
function that depends explicitly on $\phi$ and $c$.

%%%%%%%%%%%%%%%%%%%%%%%%%%%%%%%%%%%%%%%%%%%%%%%%%%%%%%%%%%%%%%%%%%%%%%%%%%%%%%%%%
%
\section{Wheeler, Boettinger, McFadden (WBM) model}

The WBM model \cite{PhysRevA.45.7424} is
characterized by the free energy density
%
\begin{equation}
  f(c,\phi)=c f_B(\phi)+(1-c)f_A(\phi)+RT[c
  \ln(c)+(1-c)\ln(1-c)].
\label{eq:fwbm}
\end{equation}
%
and a function $D_c^0(\phi,T)$ independant of $\phi$
%
\begin{equation}
  D_c^0(\phi,T)=RT
\label{eq:dconst}
\end{equation}
%
where $R$ is the universal gas constant.
Thus, we have
%
\begin{align}
  \frac{\partial f}{\partial c} & =
  f_B(\phi)-f_A(\phi)+RT\ln{\frac{c}{1-c}}
  \\
  \frac{\partial f}{\partial\phi} & =
  c\frac{\partial f_B}{\partial\phi} + (1-c)\frac{\partial
  f_A}{\partial\phi},
  \label{eq:dfdphiWBM}
  \\
  \frac{\partial^2 f}{\partial c^2} & =
  RT\frac{1}{c(1-c)},
  \label{eq:d2fdc2WBM}
  \\
  \frac{\partial^2 f}{\partial\phi\partial c}
  & = \frac{\partial f_B}{\partial\phi} -\frac{\partial
  f_A}{\partial\phi}.
  \label{eq:dfdcdphiWBM} 
\end{align}
%
Using the last two relations with (\ref{eq:ceomkks}) and (\ref{eq:dconst}),
we obtain
%
\begin{equation}
  \frac{d}{dt}c=M_c D_c^0\nabla^2 c + M_c \nabla\cdot\left(
  c(1-c)\left(\frac{\partial f_B}{\partial\phi} -\frac{\partial
  f_A}{\partial\phi}\right)\nabla\phi\right).
\label{eq:ctWBM}
\end{equation}
%
This formulation is equivalent --- at least before discretization
--- to Equation (18) of Ref. \cite{PhysRevA.45.7424}.

%%%%%%%%%%%%%%%%%%%%%%%%%%%%%%%%%%%%%%%%%%%%%%%%%%%%%%%%%%%%%%%%%%%%%%%%%%%%%%%%%
%
\section{CALPHAD model}

For the CALPHAD model of alloys, in the S and L phases, we have free energies
modeled by
%
\begin{equation}
  f^{L,S}(c,T)= f_0^{L,S}(c,T)+f_{mix}^{ideal}(c,T)+f^{L,S}_{mix}(c,T)
\end{equation}
%
where
%
\begin{equation}
  f_0^{L,S}(c,T)=c f_A^{L,S}(T)+(1-c)f_B^{L,S}(T),
\end{equation}
%
\begin{equation}
  f_{mix}^{ideal}(c,T)=
  RT[c\ln(c)+(1-c)\ln(1-c)],
\end{equation}
%
and
%
\begin{equation}
  f^{L,S}_{mix}(c,T)=c(1-c)[L_0^{L,S}(T)+L_1^{L,S}(T)(2c-1)+L_2^{L,S}(T)(2c-1)^2].
\end{equation}

For a phase value $\phi$ between 0 and 1, we can use the following model
%
\begin{equation}
  f(\phi,T)=h(\phi)f^S(T)+(1-h(\phi))f^L(T)
\end{equation}
%
and define
%
\begin{equation}
  f(c,\phi,T)=h(\phi)f^{S}(c,T)+(1-h(\phi))f^{L}(c,T)+\omega g(\phi)
\end{equation}
%
We obtain
%
\begin{equation}
\begin{split}
  f(c,\phi,T) = {} &
  c \left[h(\phi)f_A^{S}(T)+(1-h(\phi))f_A^{L}(T) \right] \\
  & + (1-c)\left[h(\phi)f_B^{S}(T)+(1-h(\phi))f_B^{L}(T) \right] \\
  & + RT[c\ln(c)+(1-c)\ln(1-c)] \\
  & + h(\phi)f^S_{mix}(c,T)+(1-h(\phi))f^{L}_{mix}(c,T) \\
  & + \omega g(\phi)
\end{split}
\end{equation}

Compared to the WBM model of the previous section, we have the additional
mixing terms
%
\begin{equation}
  h(\phi)f^S_{mix}(c)+(1-h(\phi))f^{L}_{mix}(c).
\end{equation}
%
It affects the quantities
$\partial^2 f/\partial c^2$, 
$\partial^2 f/\partial c \partial\phi$ used in particular to compute
the right hand side of the concentration time evolution equation.

Assuming that 
(\ref{eq:diffcoeff})
still holds, the time evolution equation for $c$ is given by
%
\begin{equation}
  \frac{d}{dt}c
  =\nabla D_c^0(\phi,T)\nabla c 
  + \nabla\cdot\left(
  D_c^0(\phi,T)\left(\frac{\partial^2 f}{\partial c^2}\right)^{-1}
  \frac{\partial^2 f}{\partial c \partial\phi}
  \nabla\phi\right).
\end{equation}
%
where
%
\begin{equation}
  \frac{\partial^2 f}{\partial c^2}=
  \frac{RT}{c(1-c)}
  +h(\phi)\frac{\partial^2 f^S_{mix}}{\partial c^2}
  +(1-h(\phi))\frac{\partial^2 f^L_{mix}}{\partial c^2}
\end{equation}
%
and
%
\begin{equation}
  \frac{\partial^2 f}{\partial c\partial\phi}=
  h'(\phi)
  \left[
  f_A^S-f_A^L-f_B^S+f_B^L
  +\frac{\partial f^S_{mix}}{\partial c}-\frac{\partial f^L_{mix}}{\partial c}
  \right].
\end{equation}
%
$D_c^0(\phi,T)$ can be modeled as
%
\begin{equation}
  D_c^0(\phi,T)=h(\phi) D^0_S exp(-Q^0_S/RT)
  +(1-h(\phi)) D^0_L exp(-Q^0_L/RT).
\end{equation}

%%%%%%%%%%%%%%%%%%%%%%%%%%%%%%%%%%%%%%%%%%%%%%%%%%%%%%%
%
\section{Phase variable equation}

The phase variable $\phi$ evolves in time according to
%
\begin{equation}
  \frac{d}{dt}\phi=-M_\phi\frac{\delta F}{\delta\phi} = M_\phi
  \left[\nabla\left(\frac{\partial I}{\partial\nabla\phi}\right)-
  \frac{\partial I}{\partial\phi}\right]
\label{eq:phi_evolution}
\end{equation}
%
with $I$ given by (\ref{eq:integrand}). Thus we have
%
\begin{equation}
  \frac{d}{dt}\phi=M_\phi\left(\epsilon^2\nabla\cdot\nabla\phi-
  \frac{\partial f}{\partial\phi}\right).
\label{eq:dphidt}
\end{equation}

When coupling the phase variable with a quaternion field, we
introduce an additional misorientation energy density of the form
\cite{0295-5075-71-1-131}
%
\begin{equation}
  f_{ori} = 2 H T p(\phi) | \nabla {\bf q} |.
\label{eq:fori}
\end{equation}
%
which we simply add to $f$.
Note that we use the opposite convention compared to
\cite{0295-5075-71-1-131}, that is we have replaced the polynomial
$p$ by $1-p$, and used the convention $p(0)=0$ and $p(1)=1$.

\subsection{KKS model}

For the KKS model, using Eq.(\ref{eq:dfdphi2}), we obtain
%
\begin{equation}
  \frac{d}{dt}\phi=M_\phi\left(\epsilon^2\nabla\cdot\nabla\phi
  +h'(\phi)\left(f^L(c_L)-f^S(c_S)-\mu(c_L-c_S)\right)-\omega
  g'(\phi)\right).
\label{eq:dphidt2}
\end{equation}

Adding the misorientation energy, 
the time evolution equation for $\phi$ becomes
%
\begin{equation}
\begin{split}
  \frac{d}{dt}\phi =
  M_\phi\Big\{ & \epsilon^2 \nabla^2 \phi
  + h'(\phi)\left[ f^L(c_L) - f^S(c_S) - \mu (c_L - c_S) \right]
  \\
  & - \omega g'(\phi) - 2HTp'(\phi) | \nabla {\bf q} | \Big\}
\label{eq:dphidtwithq}
\end{split}
\end{equation}
%
Note that for the particular case of one species --- no
concentration variable --- $f^S$ and $f^L$ are scalar constants
and the term $\mu(c_L-c_S)$ drops.

\subsection{CALPHAD}
The time evolution equation for $\phi$ is given by
%
\begin{equation}
\begin{split}
  \frac{d}{dt}\phi = {}
  M_\phi\Big\{ & \epsilon^2 \nabla^2 \phi
  - h'(\phi) \left[
  c (f_A^S- f_A^L) + (1 - c)(f_B^S - f_B^L)
  + f^S_{mix}(c) - f^L_{mix}(c) \right] \\
  & - \omega g'(\phi) - 2HTp'(\phi) | \nabla {\bf q} | \Big\} .
\label{eq:dphidtwithqCALPHAD}
\end{split}
\end{equation}
